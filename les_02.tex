\lesson{2}{di 08 okt 2019 10:28}{Covering spaces}
\begin{definition}
    Let $X$ be a topological space, then $X$ is simply connected iff $X$ is path connected and $\pi_1(X, x_0) = 1$ for some $x_0 \in X$
\end{definition}

\begin{remark}
    If trivial for one base point, it's trivial for all base points.
\end{remark}

\begin{eg}
    Any convex subset $C \subset \R^n$ is simply connected.
\end{eg}
\begin{eg}[Wrong proof of $\pi(S^2)$ being trivial]
    Let $f$ be a path from  $[0, 1] \to  S^2$.
    Then pick $y_0 \not\in \operatorname{Im}(f)$. Then $S^2 \setminus \{ y_0\} \approx \R^2$.
    Then use $\R^2$.

    This is wrong because we cannot always find $y_0 \not\in  \operatorname{Im}(f)$. Space filling loops! We'll see the correct proof later on.
\end{eg}

\begin{lemma}
    Suppose $X$ is \emph{simply connected} and $\alpha, \beta: I \to  X$ two paths with same start and end points.
    Then $\alpha \simeq_p \beta$.
\end{lemma}
\begin{proof}
    Simply connected implies loops are homotopic?
    Consider $\alpha * \overline{\beta} \simeq_p  e_{x_0}$, since the space is simply connected.
    \begin{align*}
        ([\alpha] * [\overline{ \beta}]) * [\beta] &= [e_{x_0}] * [\beta] = [\beta]\\
        [\alpha] * ([\overline{ \beta}] * [\beta]) &= [\alpha] * [e_{x_0}] = [\alpha]
    .\end{align*}
    And therefore $\alpha \simeq_p  \beta$.
    (Note: make sure end and start point matchs when using $*$)
\end{proof}

\section{Covering Spaces}
\begin{definition}[Evenly covered]
    Let $p : E \to  B$, surjective map (so continuous).
    Let $U \subset B$ open.
    Then $U$ is \emph{evenly covered} iff $ p^{-1}(U) = \bigcup_{\alpha \in I} V_\alpha .$ 
    with
    \begin{itemize}
        \item $V_\alpha$ open in $E$ 
        \item $V_\alpha \cap V_\beta = \O$ if $\alpha \neq \beta$
        \item  $p|_{V_\alpha} : V_\alpha \to  U$ is a homeomorphism.
    \end{itemize}
\end{definition}



\begin{remark}
    If $U' \subset U$, also open and $U$ is evenly covered, then also $U'$.
\end{remark}
\begin{definition}
    Let $p: E \to  B$ be a surjective map.
    Then $p$ is a covering projection iff $\forall b \in B, \exists  U \subset B$ open, containing $b$ such that $U$ is evenly covered by $p$. 
    Then $(E, p)$ is called a covering space.
\end{definition}
\begin{eg}
    Let $S^{1} = \{ z \in \C  \mid  |z| = 1\}$.
    Take $p: \R \to  S^{1}: t \mapsto  e^{2 \pi i t}$.
    Note that $\R$ is an easier space than $S^{1}$, and so will be $\pi_1$ ($1$ vs  $\Z$).

    There are also other covering spaces of $p$.
    For example, $p' : S^{1} \to  S^{1} : z \mapsto z^3$.

    Here we have three copies for each point. We say that the covering has $3$ sheets.
    Note that this is independent of which point we take. This is always the case!
    We can show that these are the only coverings of $S^{1}$: $\R$ and  $z \mapsto  z^{n}$.
\end{eg}

\begin{prop}
    A covering map is always a open map.
\end{prop}
\begin{proof}
    Exercise.
\end{proof}
\begin{prop}
    For any $b \in B$, $p^{-1}(b)$ is a discrete subset of $E$. (No accumulation point)
\end{prop}
\begin{proof}
    Indeed for any $\alpha \in I$, $V_\alpha \cap p^{-1}(b)$ is exactly one point.
\end{proof}

\begin{remark}
A covering is always local homeomorphism.
But: there are surjective local homeomorphism which are not covering maps.
\emph{A covering map is more than a surjective local homeomorphism}.

For example, $p: \R_0^{+} \to  S^{1}: t \mapsto  e^{2\pi i t}$. Consider the inverse image of a neighbourhood around $1$. When we restrict $p$ to the part around $0$, it is no longer a homeomorphism (we don't get the whole neighbourhood around one.)
\end{remark}

\subsection*{Creating new covering spaces out of old ones.}

\begin{itemize}
    \item Suppose $p: E \to  B$ is a covering and $B_0 \subset B$ is a subspace with the subspace topology.
        Let $E_0 = p^{-1}(B_0)$ and $p_0 = p|_{E_0}$.
        Then $(E_0, p_0)$ is a covering of $B_0$.

    \item Suppose that $(E, p)$ is a covering of $B$ and $(E', p')$ is a covering of $B'$, then $(E\times E', p\times p')$ is a covering of $B \times B'$.
\end{itemize}

\begin{eg}
    Let $T^{2} = S^{1} \times S^{1}$.

    \begin{itemize}
        \item $p: \R^2 \to  S^{1} \times S^{1}: (t, s) \mapsto  (e^{a i t}, e^{b i s})$.
        \item $p': \R \times S^{1} \to  T^{2}: (t, z) \mapsto (e^{a i t}, z^{n})$
        \item $p': S^{1} \times S^{1} \to  T^{2}: (z_1, z_2) \mapsto (z_1^{n}, z_2^{m})$
    \end{itemize}
    These are the only types of coverings of the torus. We'll prove this later on.
\end{eg}


\section{Fundamental group of the circle \emph{(and more)}}

Given $f$, when can $f$ be `lifted' to $E$?
I.e. when does there exist an $\tilde f : X \to  E$ such that $p \circ \tilde f = f$?
In this section, we'll only consider $X = [0, 1], X = [0, 1]^2$.

\begin{center}
    \begin{tikzcd}
        &E \arrow[d] \\
        X \arrow[r, "f"] \arrow[ur, "\tilde f"]& B
    \end{tikzcd}
\end{center}


\begin{lemma}[Important result]
    Suppose $(E, p)$ is a covering of $B$, $b_0 \in B$ $e_0 \in p ^{-1}(b_0)$.
    Suppose that $f: I \to B$ is a path starting at $b_0$.
    Then there exists a unique lift $\tilde f: I \to  E$ of $f$ with $\tilde f(0) =e_0$.
\end{lemma}


\begin{proof}
    For any $b$ of $B$, we choose an open $U_b$ such that $U_b$ is evenly covered by $p$.
    Then $\{ f^{-1}(U_b)  \mid  b \in B\}$ is an open cover of $I$, which is compact.
    There is a number $\delta > 0$ such that any subset of $I$ of diameter $\le  \delta$ is contained entirely in one of these opens $f^{-1}(U_b)$. (Lebesgue number lemma).
    Now, we divide the interval into pieces $0 = s_0 < s_1 < \ldots < s_n = 1$ such that $|s_{i+1} - s_i| \le \delta$.
    For any $i$, we have that $f([s_i, s_{i+1}]) \subset U_b$ for some $b$.


    We now construct $\tilde f$ by induction on $[0, s_i]$
    \begin{itemize}
        \item $\tilde f (0) = e_0$ 
        \item Assume $\tilde f$ has been defined on  $[0, s_i]$.
            Let  $U$ be an open such that $f[s_i, s_{i+1}] \subset U_b$.

            There is exactly one slice $V_\alpha$ in $p^{-1}(U_b)$ containing $\tilde f(s_i)$.
            We define  $\forall  s \in [s_i, s_{i+1}]: \tilde f(s) = (p|_{V_\alpha})^{-1} \circ f(s)$.
            By the pasting lemma, $\tilde f$ is continuous.
        \item In this way, we can construct $\tilde f$ on the whole of  $I$.
    \end{itemize}

    Uniqueness works in exactly the same way, by induction.
\end{proof}


\begin{lemma}[54.2]
    $(E, p)$ is a covering of  $B$, $b_0 \in B$, $e_0 \in E$, with $p(e_0) = b_0$.
    Suppose $F: I \times I \to  B$ is a continuous map with $f(0, 0) = b_0$, then there is a unique $\tilde F: I \times I \to E.$
    Moreover, if $F$ is a path homotopy, then also $\tilde F$ is a path homotopy.
\end{lemma}
\begin{proof}
    Same as in the one dimensional case.


    `Moreover, if $F$ is a path homotopy, then also $\tilde F$ is a path homotopy': Easy explanation in the book. Another explanation:


\begin{itemize}
    \item $F(0, t) = b_0, F(1, t) = b_1$
    \item $\tilde F (0, \cdot ) : I \to  E: t \mapsto  \tilde F(0, t)$ is a path starting at $e_0$,
        and is a lift of $F(0, \cdot )$ (the constant path at $b_0$).

        The path $k: I \to  E: t \mapsto  e_0$ is also a lift of the constant path at $b_0$ starting at $e_0$.
        Then $k = \tilde F(0, \cdot )$, as the lift is unique.
\end{itemize}
\end{proof}


\begin{theorem}[54.3]
    Let $(E, p)$ be a covering of $B$, $b_0 \in B$, $e_0 \in E$ with $p(e_0) = b_0$.
    Let $f, g$ be two paths in $B$ starting in $b_0$ s.t. $f \simeq_p  g$ (so $f$ and $g$ end at the same point).
    Let $\tilde f, \tilde g$ be the unique lifts of $f, g$ starting at $e_0$.
    Then $\tilde f \simeq_p  \tilde g$, and so $\tilde f(1) = \tilde g(1)$.
\end{theorem}
\begin{proof}
    $F: I\times I \to  B$ is a path homotopy between $f$ and $g$.
    Then $\tilde F: I \times I \to  E$ with $\tilde F( 0, 0) = e_0$.
    Then $\tilde F$ is a path homotopy, by the previous result, between $\tilde F(\cdot , 0)$ and $\tilde F(\cdot , 1)$.
    Note that $p  \circ  \tilde F(t, 0) = F(t, 0) = f(t)$ and $p  \circ  \tilde F (t, 1) = F(t, 1) = g(t)$.
    By uniqueness $\tilde F(\cdot, 0) = \tilde f$, $\tilde F(\cdot , 1) = \tilde g$.

\end{proof}


We've shown that homotopy from below lifts to above.  The converse is easy.
Now we're ready to discuss the relation between groups and covering spaces.

\begin{definition}
    Let $(E, p)$ be a covering of $B$. $b_0 \in B$, $e_0 \in E$ and $p(e_0) = b_0$.
    Then the lifting correspondence is the map  
    \begin{align*}
        \phi: \pi(B, b_0) &\longrightarrow p^{-1}(b_0) \\
        [f]&\longmapsto \tilde f(1), \text{where $\tilde f$ is the unique lift of $f$, starting at $e_0$}
    .\end{align*}
    This is well-defined because $[f] = [g] \implies \tilde f \simeq_p  \tilde g \implies \tilde f(1) = \tilde g(1)$.
    This $\phi$ depends on the choice of $e_0$.
\end{definition}

\begin{figure}[ht]
    \centering
    \incfig{lifting-correspondence}
    \caption{Lifting Correspondence}
    \label{fig:lifting-correspondence}
\end{figure}

\begin{theorem}[54.4]
    If $E$ is path connected, then $\phi$ is a surjective map.
    If $E$ is simply connected, then $\phi$ is a bijective map.
\end{theorem}
\begin{proof}
    Suppose $E$ is path connected, and let $e_0, e_1 \in p ^{-1} (b_0)$.
    Consider a path $\tilde f: I \to  E$ with $\tilde f(0) = e_0$ and $\tilde f(1) = e_1$. This is possible because $E$ is path connected.
    Let $f = p  \circ \tilde f: I \to B$ with $f(0) = p(e_0) = b_0$ and $f(1) = p(e_1) = b_0$, so $f$ is a loop based at  $b_0$.
    So $f$ is a loop at $b_0$ and its unique lift to $E$ starting at $e_0$ is $\tilde f$.
    Hence  $\phi[f] = \tilde f(1) = e_1$,
    which shows that $\phi$ is surjective.

    Now assume that $E$ is simply connected (group is trivial).
    Consider $[f], [g] \in \pi(B_0)$ with $\phi[f] = \phi[g]$.
    This implies  $\tilde f(1) = \tilde g(1)$.
    These start at $e_0$.
    It follows from Lemma 52.3 that $\tilde f \simeq_p \tilde g$.
\end{proof}

\begin{eg}
    Take the circle and the real line as covering space. Then $p ^{-1}(1) = \Z$.
    So we know that as a set $\pi(S^{1})$ is countable.
    Therefore, $p  \circ  \tilde f \simeq_p p  \circ  \tilde g $.
    This implies that $f \simeq_p  g$, and therefore $[f] = [g]$.
\end{eg}

% Make inline math better, non recursive.


\begin{theorem}
    $\pi_1(S^{1}, 1) \cong (\Z, +)$.
\end{theorem}
\begin{proof}
    Take $b_0 = 1$ and $e_0=0$ and $p: \R \to S^{1}: t \mapsto e^{2 \pi i t}$.
    Then $p ^{-1}(b_0) = \Z$.
    And since, $\R$ is simply connected, we have that $\phi: \pi(S, 1) \to  \Z: [f] \mapsto  \tilde f(1)$ is a bijection.

    Now we'll show that it's a morphism.
    Let $[f]$ and $[g]$ elements of the fundamental group of $S^1$ and assume that $\phi[f] = \tilde f(1) = m$ and $\phi[g] = \tilde g(1) = n$.

    We're going to prove that $\phi([f]*[g]) = \phi([f]) + \phi([g]) = m + n$. 
    Define $\tilde{\tilde g}: I \to  \R: t \mapsto  \tilde g(t) + m$.
    Then $p \circ \tilde{\tilde g} = p  \circ  \tilde g = g$, as $p(s+m) = p(s)$ for all  $m$.
    Now, look at $\tilde f * \tilde{\tilde g}$.
    This is a lift of $p  \circ (\tilde f * \tilde{\tilde g}) = (p  \circ  \tilde f) * (p  \circ  \tilde{\tilde{g}}) = f*g$, which starts at $0$.
    Hence, $\phi([f]*[g]) = \phi([f*g]) = $ the end point of  $\tilde f * \tilde{\tilde g}$, so  $\tilde{\tilde{g}}(1) = \tilde g(1) + m = n + m$.
\end{proof}

The following lemma makes the fact that the covering space is simpler than the space itself exact.

\begin{lemma}[54.6]
    Let $(E, p)$ be a covering of $B$. $b_0 \in B$, $e_0 \in E$ and $p(e_0) = b_0$. Then
    \begin{enumerate}
        \item $p_*: \pi(E, e_0) \to  \pi(B, b_0)$ is a monomorphism (injective).
            
        \item[2!] Let $H = p_*(\pi_1(E, e_0))$. The lifting correspondence induces a well defined map\footnote{different notation than the book}
            \begin{align*}
                \Phi: \mfaktor{H}{\pi_1(B, b_0)} &\longrightarrow  p ^{-1}(b_0) \\
                H * [f]& \longmapsto \phi[f]
            ,\end{align*}
            so $\phi$ is constant on right cosets.
            Dividing by $H$ makes $\Phi$ always bijective, even when $E$ is not simply connected.
        \item[3.] Let $f$ be a loop based at $b_0$, then $\tilde f$ is a loop at $e_0$ iff $[f] \in H$.
    \end{enumerate}
\end{lemma}
\begin{proof}
    \begin{enumerate}
        \item Let $\tilde f: I \to  E$ be a loop at $e_0$ and assume that $p_*[\tilde f] = 1$. (Then we'd like to show that $f$ itself is trivial.)
            This implies $p  \circ  \tilde f \simeq_p  e_{b_0}$.
            This implies that $\tilde f \simeq_p \tilde e_{b_0} = e_{e_0}$, or $[\tilde f] = 1$.
        \item We have to prove two things: 
            \begin{description}
                \item [Well defined] $H*[f] = H*[g] \implies \phi(f) = \phi(g)$.

                    Assume $[f] \in H * [g]$, or $H*[f] = H*[g]$.
                    This means that  $[f] = [h] * [g]$, were  $h = p  \circ  \tilde h$ for some loop $\tilde h$ at $e_0$.
                    In other words $[f] = [h*g]$, or  $f \simeq_p  h*g$.
                    Let $\tilde f$ be the unique lift of $f$ starting at $e_0$.
                    Let $\tilde g$ be the unique lift of  $g$ starting at $e_0.$
                    Then $\tilde h * \tilde g$ (which is allowed, $\tilde h$ is a loop) the unique lift of $h * g$ starting at $e_0$.


                    $\tilde f(1) = \phi(f) = \phi(h * g) = (\tilde h * \tilde g)(1) = \tilde g(1) = \phi(g)$.
                    If the cosets are the same, then the end points of the lifts are also the same.
                \item [Injective]$H*[f] = H*[g] \impliedby \phi(f) = \phi(g)$.
                    The end points of $f$ and $g$ are the same

                    Now consider $\tilde h = \tilde f * \overline{\tilde g}$.
                    Then $[ \tilde h]*[\tilde g] = [\tilde f] * [\overline{\tilde g}] * [\tilde g] = [\tilde f]$.
                    By applying $p_*$,  $[h]*[g] = [f]$.
            \end{description}
        \item Trivial. Exercise. Apply $2.$ with the constant path.
    \end{enumerate}
\end{proof}
\begin{remark}
    $k: X \to  Y$ induces a morphism $k_*$, we've proved that earlier. Here we only showed injectiveness.
\end{remark}

