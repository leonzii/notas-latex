%\correct{correccion} {Asi es uwu}

% This is for an intro
\begin{itemize}
    \item Lineas de separacion con /hr
        \begin{itemize}
            \item item en item
        \end{itemize}
    \hr
    
    \item Second part: Homology groups, via cudi, chapter of a book.
    \item 10 problems (solve them during the semester) No feedback during the semester, but asking questions is allowed. Working together is allowed.
    \item Exam (completely open book) First 4 questions: 1h30 prep. Last one 30 min, no prep
        \begin{enumerate}
            \item Theoretical question (open book, explain the proof, \ldots) \hfill /4
            \item New problem (comparable to one of the 10 problems)\hfill /4
            \item Explain your solution $n$-th problem solved at home \hfill /4
            \item Explain your solution $m$-th problem solved at home\hfill /4
            \item 4 small questions \hfill /4
        \end{enumerate}
        After the exam, hand in your solutions of the other problems. After a quick look, the points of 3.\ and 4.\ can be $\pm 1$ (in extreme cases $\pm 2$)
    \item No exercise classes for this course!
\end{itemize}

% Chapter 0 (usually an introduction)
\setcounter{chapter}{-1}
\chapter{Introduction}

\section{What is algebraic topology?}
Functor from category of topological spaces to the category of groups.

Two systems we'll discuss:
\begin{itemize}
    \item fundamental groups
    \item homology groups
\end{itemize}


\begin{eg}
    Suppose we have a functor.
    If $G_X \not\cong G_Y$, then  $X$ and  $Y$ are not homeomorphic.
    If `shadows' are different, then objects themselves are different too.
\end{eg}

\begin{explanation}
    Suppose $X$ and $Y$ are homeomorphic.
    Then $\exists f: X \to  Y$ and $g: Y \to  X$, maps (maps are always continuous in this course), such that $g  \circ f = 1_X$ and $f \circ g = 1_Y$.
    Then $f_*: G_X \to G_Y$ and $g_*: G_Y \to  G_X$ such that $(g \circ f)_* = (1_X)_*$ and  $(f \circ g)_* = (1_Y)_*$. Using the rules discussed previously, we get
    \[
    g_*  \circ f_* = 1_{G_X} \quad f_*  \circ  g_* = 1_{G_Y}
    ,\] 
    which means that $f_* : G_X \to  G_Y$ is an isomorphism.
\end{explanation}

\section{Fundamental group}
Pick a base point $x_0$ and consider it fixed. (The fundamental group will not depend on it. We assume all spaces are path connected)
$X \leadsto \pi(X)$.
\begin{itemize}
    \item A loop based at $x_0 \in X$ is a map $f: I = [0, 1] \to X$, $f(0) = f(1) = x_0$.
    \item Loops are equivalent if one can be deformed in the other in a continuous way, with the base point fixed.
    \item The fundamental group consists of equivalent classes of loops.
\end{itemize}

\begin{eg}
    Let $X = B^2$ (2 dimensional disk).
    Then $\pi(B^2) = 1$, because every loop is equivalent to the `constant' loop.
\end{eg}
\begin{eg}
    Let $X = S^{1}$ and pick $x_0$ on the circle.
    Two options: 
    \begin{itemize}
        \item The loop is trivial equivalent to the constant loop
        \item The loop goes around the circle.
        \item The loop goes around the circle, twice.
        \item The loop goes around the circle, clockwise, once
        \item \ldots
    \end{itemize}
    $\pi(S^{1}) \cong \Z$ (proof will follow).
\end{eg}

The composition of loops is simply pasting them.
In the case of the circle, the loop $-1$  $\circ$ the loop  $2$ is the loop $1$.

Suppose $\alpha: I \to  X$ and $f : X\to Y$. Then we define 
\[
    f_*[\alpha] = [f \circ \alpha]
.\] 


\begin{theorem}[Fixed point theorem of Brouwer]
    Any continuous map from a rectangle to itself has at least one fixed point.
\end{theorem}
\begin{proof}
    Suppose there is no fixed point, so $f(x) \neq x$ for all  $x \in B^2$.
    Then we can construct map $r : B^2 \to  S_1$ as follows:
    take the intersection of the boundary and half ray between $f(x)$ and $x$.

    If $x$ lies on the boundary, we have the identity map.
    This map is continuous.
    Then we have $S^1 \to B^2 \to S^1$, via the inclusion and $r$.
    Looking at the fundamental groups:
    \[
        \pi(S^1)  =\Z \to  \pi(B^2) = 1 \to  \pi(S^1) = \Z
    .\] 
    The map from $\pi(S^1) \to  \pi(S^{1})$ is the identity map, but the first map maps everything on $1$. \phantom\qedhere\hfill\contra 
\end{proof}



\setcounter{chapter}{8}
\chapter{Fundamental group}

\setcounter{section}{50}
\section{Homotopy of paths}
\begin{definition}[Homotopy]
    Let $f,g:X\to Y$ be maps (so continuous). Then a homotopy between $f$ and $g$ is a continuous map $H: X\times I \to Y$ such that
    \begin{itemize}
        \item $H(x, 0) = f(x)$, $H(x, 1) = g(x)$ 
        \item For all $t \in I$, define $f_t: X \to  Y: x \mapsto  H(x, t)$
    \end{itemize}
    We say that $f$ is homotopic with $g$, we write $f \simeq g$.
    If $g$ is a constant map, we say that $f$ is null homotopic.
\end{definition}

\begin{definition}[Path homotopy]
    Let $f, g: I \to  X$ be two paths such that $f(0) = g(0) = x_0$ and $f(1) = g(1) = x_1$. Then $H: I \times  I \to  X$ is a path homotopy between $f$ and $g$, if and only if
    \begin{itemize}
        \item $H(s,0) = f(s)$ and  $H(s, 1) = g(s)$ (homotopy between maps)
        \item  $H(0, t) = x_0$ and $H(1, t) = x_1$ (start and end points fixed)
    \end{itemize}
    Notation: $f \simeq_p g$.
\end{definition}


\begin{lemma}
    $\simeq$ and  $\simeq_p$ are equivalence relations.
\end{lemma}
\begin{proof}
    \leavevmode
    \begin{itemize}
        \item Reflective: $F(x, t) = f(x)$
        \item Symmetric: $G(x, t) = H(x, 1-t)$ 
        \item Transitive:  Suppose $f \simeq g$ and  $g \simeq h$, with  $H_1, H_2$ resp.
            \[
                H(x, t) = \begin{cases}
                    H_1(x, 2t) & 0 \le t \le  \frac{1}{2}\\
                    H_2(x, 2t-1) & \frac{1}{2} \le  t \le  1
                \end{cases}
            . \qedhere
        \] 
    \end{itemize}
\end{proof}


\begin{eg}[Trivial, but important]
   Let $C \subset \R^n$ be a convex subset.
   \begin{itemize}
       \item Any two maps $f, g: X \to C$ are homotopic.
       \item Any two paths $f, g: I \to  C$ with $f(0) = g(0)$ and  $g(1) = f(1)$ are path homotopic.
   \end{itemize}

   Choose $H: X \times I \to  C: (x, t) \to  H(x, t) = (1-t) f(x) + t g(x)$.
\end{eg}


\subsection*{Product of paths}
Let $f: I\to X$, $g: I \to X$ be paths, $f(1) = g(0)$.
Define  \[
    f*g: I \to X: s \mapsto \begin{cases}
        f(2s) & 0 \le  s \le  \frac{1}{2}\\
        g(2s - 1) & \frac{1}{2}\le s\le 1.
\end{cases}
\]
\begin{remark}
    If $f$ is path homotopic to $f'$ and $g$ path homotopic to $g'$ (which means that $f(1) = f'(1) = g(0) = g'(0)$), then $f * g \simeq_p f' * g'$.
\end{remark}

So we can define $[f] * [g] := [f*g]$ with $[f] := \{g : I \to  X  \mid g \simeq_p f\} $

\begin{theorem}\leavevmode
    \begin{enumerate}
        \item $[f] * ([g] * [h])$  is defined iff $([f] * [g])*[h]$ is defined and in that case, they are equal.
        \item Let $e_x$ denote the constant path $e_x: I \to  X: s \mapsto x$, $x \in X$. If $f(0) = x_0$ and $f(1) = x_1$ then $[e_{x_0}] * [f] = [f]$ and $[f] * [e_{x_1}] = [f]$.
        \item Let $\overline{f} : I \to  X: s \mapsto  f(1-s)$. Then $[f] * [\overline{f}] = [e_{x_0}]$ and $[\overline{f}]*[f] = [e_{x_1}]$
    \end{enumerate}
\end{theorem}
\begin{proof}
    First, two observations
    \begin{itemize}
        \item Suppose $f\simeq_p g$ via homotopy $H$, $f, g: I\to X$.
            Let $k: X \to  Y$.
            Then $k  \circ  f \simeq_p k \circ g$ using $k  \circ H$.
        \item If $f*g$ (not necessarily path homotopic). Then $ k  \circ (f*g) = (k \circ f) * (k \circ g) .$ 
    \end{itemize}

    Now, the proof
    \begin{itemize}
        \item[2.]  Take $e_0: I \to  I: s \mapsto 0$.
            Take $i: I \to  I: s \mapsto s$ 
            Then $e_0*i$ is a path from $0$ to $1 \in I$.
            The path $i$ is also such a path.
            Because $I$ is a convex subset, $e_0*i$ and $i$ are path homotopic, $e_0 *i\simeq_p i$.
            Using one of our observations, we find that
            \begin{align*}
                f  \circ (e_0 * i) &\simeq_p  f  \circ i\\
                (f  \circ  e_0) * (f  \circ  i ) &\simeq_p f\\
                e_{x_0} * f  &\simeq_p f\\
                [e_{x_0}] * [f]  &= [f]
            .\end{align*}

        \item [3.] Note that $i * \overline{i} \simeq_p e_0$.
            Now, applying the same rules, we get 
            \begin{align*}
                f  \circ (i * \overline{i}) &\simeq_p f  \circ  e_0\\
                f * \overline{f} &\simeq_p  e_{x_0}\\
                [f] * [\overline{f}] & = [e_{x_0}]
            .\end{align*}
        \item[1.]  Remark: Only defined if $f(1) = g(0), g(1) = h(0)$.
            Note that $f*(g*h) \neq (f*g)*h$. The trajectory is the same, but the speed is not.

            Assume the product is defined.
            Suppose $[a, b]$,  $[c, d]$ are intervals in  $\R$.
            Then there is a unique positive (positive slope), linear map from $[a, b] \to [c,d]$.
            For any $a, b \in [0, 1)$ with $ 0<a<b<1$, we define a path
            \begin{align*}
                k_{a,b}: [0, 1] &\longrightarrow  X\\
                [0, a] & \xrightarrow{\text{lin.}}[0, 1] \xrightarrow{f}  X\\
                [a, b] & \xrightarrow{\text{lin.}}[0, 1] \xrightarrow{g}  X\\
                [b, 0] & \xrightarrow{\text{lin.}}[0, 1] \xrightarrow{h}  X\\
            \end{align*} 

            Then $f*(g*h) = k_{\frac{1}{2}, \frac{3}{4}}$ and $(f*g)*h = k_{\frac{1}{4}, \frac{1}{2}}$

            Let $\gamma$ be that path $\gamma:I \to  I$ with the following graphs:

            Note  that $\gamma \simeq_p i$.
            Now, using the fact that composition of positive linear maps is positive linear.
            \begin{align*}
                k_{c, d}  \circ  \gamma &\simeq_p  k_{c, d}  \circ i\\
                k_{a, b} &\simeq_p  k_{c, d},
            \end{align*} 
            which is what we wanted to show.
    \end{itemize}
\end{proof}


\section{Fundamental group}
\begin{definition}
    Let $X$ be a space and $x_0 \in X$, then the fundamental group of $X$ based at $x_0$ is
    \[
        \pi(X, x_0) = \{ [f]  \mid  f : I \to X, f(0) = f(1) = x_0\} 
    .\] 
    (Also $\pi_1(X, x_0)$ is used, first homotopy group of $X$ based at $x_0$)

    For $[f], [g] \in \pi(X, x_0)$, $[f] * [g]$ is always defined,  $[e_{x_0}]$ is an identity element, $*$ is associative and $[f]^{-1} = [\overline{f}]$. This makes $(\pi(X, x_0), *)$ a group.
\end{definition}
\begin{eg}
    If $C \subset \R^n$, convex then $\pi(X, x_0) = 1$.
    E.g. $\pi(B^2, x_0) = 1$.
\end{eg}

\begin{remark}
    All groups are a fundamental group of some space.
\end{remark}

Question: how does the group depend on  the base point?

\begin{theorem}[52.1]
    Let $X$ be a space, $x_0, x_1 \in X$ and $\alpha: I \to X$ a path from  $x_0$ to $x_1$.
    Then
    \begin{align*}
        \hat{\alpha}: \pi(X, x_0) &\longrightarrow\pi(x, x_1)\\
        [f] &\longmapsto  [\overline{\alpha}] * [f] * [\alpha]
    .\end{align*}
    is an isomorphisms of groups. Note however that this isomorphism depends on $\alpha$.
\end{theorem}
\begin{proof}
    Let $[f], [g] \in \pi_1(X, x_0)$.
    Then 
    \begin{align*}
        \hat{\alpha}([f]*[g]) &= [\overline{\alpha}] * [f] * [g] * [\alpha]\\
                              &= [\overline{\alpha}] *[f] * [\alpha] * [\overline{\alpha}] * [g] * [\alpha]\\
                              &= \hat{\alpha}[f] * \hat{\alpha}[g]
    .\end{align*}
    We can also construct the inverse, proving that these groups are isomorphic.
\end{proof}


\begin{remark}
    If $f: (x, x_0) \to (Y, y_0)$ is a map of pointed topology spaces ($f: X \to Y$ continuous and $f(x_0) = y_0$).
    Then \[
        f_*:\pi(X, x_0) \to  \pi(Y, y_0): [\gamma] \mapsto [f  \circ \gamma]
    \]
    is a morphism of groups, because of the two `rules' discussed previously, with
    \[
        (f \circ g)_* = f_*  \circ g_* \qquad (1_X)_* = 1_{\pi(X, x_0)}
    .\] 
\end{remark}
