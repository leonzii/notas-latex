% This is for an intro
\textbf{Para los preliminares al inicio del curso!}
\begin{itemize}
    \item Lineas de separacion con /hr
        \begin{itemize}
            \item item en item
        \end{itemize}
    \hr

    \item Second part: Homology groups, via cudi, chapter of a book.
    \item 10 problems (solve them during the semester) No feedback during the semester, but asking questions is allowed. Working together is allowed.
    \item Exam (completely open book) First 4 questions: 1h30 prep. Last one 30 min, no prep
        \begin{enumerate}
            \item Theoretical question (open book, explain the proof, \ldots) \hfill /4
            \item New problem (comparable to one of the 10 problems)\hfill /4
            \item Explain your solution $n$-th problem solved at home \hfill /4
            \item Explain your solution $m$-th problem solved at home\hfill /4
            \item 4 small questions \hfill /4
        \end{enumerate}
        After the exam, hand in your solutions of the other problems. After a quick look, the points of 3.\ and 4.\ can be $\pm 1$ (in extreme cases $\pm 2$)
    \item No exercise classes for this course!
\end{itemize}

% Chapter 0 (usually an introduction)
\setcounter{chapter}{-1}
\chapter{Introduction}

\section{Demos de los diversos ambientes (enviroments)}


Usamos una variacion del tema catpucchin.



\begin{deff}
  Usamos \textbf{definition} para las definiciones.
\end{deff}

\begin{eg}
  Usamos \textbf{eg}, para los ejemplos.
\end{eg}


\begin{prop}
  Usamos \textbf{prop}, para las proposiciones.
\end{prop}

\begin{proof}
  Usamos \textbf{proof}, para las pruebas, estas se pegan a la proposicion
  anterior.
\end{proof}

\begin{lemma}
	Tenemos tambien \textbf{lemma}.
\end{lemma}

\begin{cor}
	Y \textbf{cor}.
\end{cor}


\section{Notas y }

\begin{remark}
	Usamos \textbf{remark} para comentarios.
\end{remark}

\begin{note}
	Usamos \textbf{note} para notas.
\end{note}

\begin{notation}
	Usamos \textbf{note} para notacion.
\end{notation}


\begin{previouslyseen}
  Usamos \textbf{previouslyseen} para algo que ya vimos.
\end{previouslyseen}

\begin{problem}
  Usamos \textbf{problem} para denotar problemas.
\end{problem}

\begin{observe}
  Usamos \textbf{observe} para observar, xD.
\end{observe}

\begin{property}
  Usamos \textbf{property} para hacer referencia a una propiedad.
\end{property}

\begin{intuition}
  Usamos \textbf{intuition} para dar una intuicion.
\end{intuition}

\subsection{Code}
\begin{lstlisting}[language=bash, style=mystyle]
  # An example of bash code
	sudo sed -i '/^background=/c\background=~/.lock.png' \
	 /etc/lightdm/slick-greeter.conf
\end{lstlisting}

\section{Notacion}
\begin{itemize}
  \item \textbf{\textbackslash{N}} para \N.
  \item \textbf{\textbackslash{R}} para \R.
  \item \textbf{\textbackslash{Z}} para \Z.
  \item \textbf{\textbackslash{O}} para \O.
  \item \textbf{\textbackslash{Q}} para \Q.
  \item \textbf{\textbackslash{implies}} para $\implies$.
  \item \textbf{\textbackslash{impliedby}} para $\impliedby$.
  \item \textbf{\textbackslash{iff}} para $\iff$.
  \item \textbf{\textbackslash{epsilon}} para $\epsilon$.
\end{itemize}


\section{Ejercicios}
\ejercicio{tema a tratar} Para ejercicios!


\end{noot}
% Podemos cambiar la numeracion facilmente!
% \setcounter{chapter}{8}
% \chapter{Fundamental group}
% \setcounter{section}{50}
